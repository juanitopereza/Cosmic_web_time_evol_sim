\documentclass{article}
\usepackage[utf8]{inputenc}
\usepackage{amsmath}

\title{Caracterización de los elementos estructurales de la cosmic web y análisis de su evolución temporal a partir de datos generados por una simulación cosmológica de N-cuerpos}
\author{Juan Sebastián Pérez\\ \\Director: Jaime Forero}
\date{Noviembre 2015}

\begin{document}

\maketitle

\section{Introducción}

Las simulaciones cosmológicas han sido de gran utilidad en 
las últimas décadas para predecir la distribuci\'on de materia 
en el universo a grandes escalas.
Los resultados des simulaciones pueden ser comparados
directamente con las observaciones astron\'omicas de la distribuci\'on
de galaxias. 
Tanto las observaciones astronómicas como las simulaciones muestran
una estructura cosmológica bien definida la cual se conoce como
“cosmic web”, la cual cuenta con elementos 
estructurales definidos según sus características geométricas como
vacíos, filamentos, hojas y nudos.  
Así pues, el estudio de las  características morfológicas de esta red
son un tema de estudio de alto impacto en la astrofísica moderna. 

En este trabajo se har\'a una caracterización de la red c\'osmica a
en simulaciones cosmológicas de N-cuerpos. 
El objetivo principal es el estudio de la evoluci\'on de la red
c\'osmica en escalas de tiempo cosmol\'ogicas.
La simulación cosmol\'ogica se hará con el c\'odigo Gadget [REF], 
un estándar para la realización de simulaciones cosmológicas. GADGET
est\'a desarrollado para trabajar en sistemas de cómputo de múltiples
procesadores en paralelo, 


Los datos obtenidos en la simulación se analizar\'an para caracterizar
la red c\'osmica en el volumen simulado. 
El an\'alisis  utiliza los valores propios del Hessiano del potencial
gravitacional $\phi$ para clasificar localmente cada lugar de la
simulaci\'on como un vac\'io, filamento, hoja o nudo.
[ESTE CODIGO TAMBIEN ES PUBLICO \verb"https://github.com/astroandes/TV-web"]

<<<<<<< HEAD
Teniendo en cuenta lo anterior, en este trabajo pretendo realizar una caracterización de los elementos estructurales de la cosmic web nombrados anteriormente, a partir del análisis de los datos obtenidos al realizar una simulación cosmológica de N-cuerpos. 

Para crear la simulación, haré uso del código GADGET-2\cite{Gadget}, creado por el profesor Dr. Volker Springel de la universidad de Heildeberg, Alemania. Este código, escrito en C, se ha convertido en un estándar para la realización de simulaciones cosmológicas y fue desarrollado para trabajar en sistemas de cómputo de múltiples procesadores en paralelo, lo cual permite optimizar el desempeño del programa para así lograr simulaciones con un número de partículas considerable.

Posteriormente, usaré los datos obtenidos a partir de la simulación para realizar un análisis de las características geométricas de cada punto espacial con el fin de categorizarlos en 4 grupos: vacíos, filamentos, hojas y nudos. Para realizar esta categorización, haré un estudio de los valores propios del tensor de deformación $T_{\alpha\beta}$, el cual corresponde al Hessiano del potencial gravitacional $\phi$\cite{Forero-Romero}\cite{tv-web}. Para realizar la clasificación, se diagonalizará el Hessiano del potencial gravitacional evaluado en cada punto espacial y se hará una cuenta de los valores propios obtenidos por encima de cierto umbral; si el número de valores propios obtenidos es cero, el punto corresponderá a un vacío, si el número es uno corresponderá a un filamento, si es dos será una hoja y si es tres será un nudo.    

\section{Objetivo General}

Realizar una caracterización de los elementos estructurales de la cosmic web (vacíos, filamentos, hojas y nudos) y su evolución temporal a partir de los datos generados por una simulación cosmológica realizada con el código GADGET-2.
=======

\section{Objetivo General}

Realizar una caracterización de los elementos estructurales de la
cosmic web (vacíos, filamentos, hojas y nudos) y su evolución temporal
a partir de los datos generados por una simulación cosmológica
realizada con el código GADGET2. 
>>>>>>> 997183407a8e75101fc71e1a3df8ebfe8265dcf6

\section{Objetivos Específicos}

\begin{enumerate}

<<<<<<< HEAD
\item Familiarizarse con el uso de máquinas de computo de múltiples procesadores en paralelo. 
\item Aprender a compilar y correr el código GADGET-2 para simulaciones de N-cuerpos.
\item Realizar una simulación cosmológica, usando los parámetros cosmológicos definidos por las últimas observaciones astronómicas y usando el código GADGET-2.
\item Evaluar el Hessiano del potencial gravitacional para cada punto espacial y en cada paso temporal y encontrar los valores propios que se encuentran por encima de un cierto umbral.
\item Clasificar cada punto espacial, dependiendo de los valores propios obtenidos, como: vacíos, filamentos, hojas o nudos.
\item Analizar la evolución de la estructura observada en cada paso temporal.

=======
\item Familiarizarse con el uso de máquinas de c\'omputo de múltiples
  procesadores en paralelo.  
\item Aprender a compilar y correr el código GADGET2 para simulaciones
  de N-cuerpos. 
\item Realizar una simulación cosmológica, usando los parámetros
  cosmológicos definidos por las últimas observaciones astronómicas y
  usando el código GADGET2. 
\item Evaluar el Hessiano del potencial gravitacional para cada punto
  espacial y en cada paso temporal y encontrar los valores propios que
  se encuentran por encima de un cierto umbral. 
\item Clasificar cada punto espacial, dependiendo de los valores
  propios obtenidos, como: vacíos, filamentos, hojas o nudos. 
\item Analizar la evolución de la estructura observada en cada paso
  temporal. 
>>>>>>> 997183407a8e75101fc71e1a3df8ebfe8265dcf6
\end{enumerate}


\section{Metodología}

<<<<<<< HEAD
Esta será una monografía computacional en la cuál usaré el código GADGET-2 que ha sido desarrollado  para correr simulaciones cosmológicas en máquinas de múltiples procesadores paralelos, por lo tanto, es necesario tener acceso al cluster de la universidad como primera medida. Adicionalmente, y aunque no es indispensable, requeriré acceso al laboratorio computacional de ciencias ubicado en el salón Q406.

La primera etapa del trabajo se centrará en familiarizarme con las herramientas tecnológicas necesarias para la elaboración de la simulación. Lo anterior corresponderá a aprender acerca del manejo de sistemas de cómputo de múltiples procesadores en paralelo, en específico, obtener acceso al cluster de la universidad y a manejar su ambiente de programación. Adicionalmente, en esta etapa pretendo obtener un conocimiento básico del funcionamiento del código GADGET-2 que será usado para la simulación cosmológica a partir de la cual se generarán los datos. Para familiarizarme con el programa, correré simulaciones sencillas las cuales vienen incluidas en la documentación del código, tales como colisiones de galaxias, colapso de nubes de gas, entre otras. 

Posteriormente, y una vez halla obtenido un entendimiento adecuado del funcionamiento de GADGET-2, procederé a realizar la simulación cosmológica necesaria para el análisis estructural de los elementos de la cosmic web. Como se mencionó anteriormente, el análisis consistirá en calcular los valores propios del Hessiano del potencial gravitacional evaluado en cada punto espacial con el fin de determinar las características geométricas de cada uno de ellos y así clasificarlos como vacíos, filamentos, hojas o nudos.

Finalmente, realizaré el análisis comentado anteriormente para cada paso temporal con propósito de tener una imagen de la evolución temporal de estas estructuras y de su aporte a la morfológica de la cosmic web. En esta última etapa haré comparaciones con datos de otras simulaciones y de observaciones astronómicas con el fin de corroborar que la estructura obtenida en este estudio corresponde con la estructura cosmológica observada.         
=======
Esta será una monografía computacional. Utilizar\'e el código
GADGET2 qpara correr simulaciones cosmológicas en máquinas de
múltiples procesadores paralelos, por lo tanto, es necesario tener
acceso al cluster de la universidad como primera
medida. Adicionalmente, y aunque no es indispensable, se requerirá
acceso al laboratorio computacional de ciencias ubicado enel salón
Q406.  
  
La primera etapa del trabajo se centrará en familiarizarse con las
herramientas tecnológicas necesarias para la elaboración de la
simulación. Lo anterior corresponderá a familiarizarse con el manejo
de sistemas de cómputo de múltiples procesadores en paralelo, en
específico, obtener acceso al cluster de la universidad y a manejar su
ambiente de programación. Adicionalmente, en esta etapa se pretende
obtener un conocimiento básico del funcionamiento del código GADGET2
que será usado para la simulación cosmológica a partir de la cual se
generarán los datos. Para familiarizarse con el programa, se harán
simulaciones sencillas las cuales vienen incluidas en la documentación
del código, tales como colisiones de galaxias, colapso de nubes de
gas, entre otras.  

Posteriormente, y una vez se haya obtenido un entendimiento adecuado
del funcionamiento de GADGET2, se procederá a realizar la simulación
cosmológica necesaria para el análisis estructural de los elementos de
la cosmic web. Como se mencionó anteriormente, el análisis consistirá
en calcular los valores propios del Hessiano del potencial
gravitacional evaluado en cada punto espacial con el fin de determinar
las características geométricas de cada uno de ellos y así
clasificarlos como vacíos, filamentos, hojas o nudos. 

Finalmente, se realizará el análisis comentado anteriormente para cada
paso temporal con propósito de tener una imagen de la evolución
temporal de estas estructuras y de su aporte a la morfológica de la
cosmic web. En esta última etapa se realizarán comparaciones con datos
de otras simulaciones y de observaciones astronómicas con el fin de
corroborar que la estructura obtenida en este estudio corresponde con
la estructura cosmológica observada.          
>>>>>>> 997183407a8e75101fc71e1a3df8ebfe8265dcf6


\section{Cronograma}

Tarea 1: Revisión inicial de la literatura \\
Tarea 2: Familiarización con el manejo del cluster y primeros acercamientos al uso de GADGET-2 \\
Tarea 3: Correr las simulaciones de prueba que vienen incluidas en la documentación de GADGET-2  \\
Tarea 4: Investigar acerca de los parámetros que se usarán para la elaboración de la simulación que generará los datos \\
Tarea 5: Realizar la simulación mencionada anteriormente \\
Tarea 6: Analizar preliminarmente los datos obtenidos a partir de la simulación \\
Tarea 7: Categorizar cada punto espacial de la simulación realizada según sus características geométricas en vacíos, filamentos, hojas o nudos (Diagonalización de $T_{\alpha\beta}$). \\
Tarea 8: Estudiar la evolución temporal de las características geométricas de cada punto espacial. \\
Tarea 9: Interpretar los resultados obtenidos de la caracterización para obtener así información acerca de la estructura morfológica de la cosmic web \\
Tarea 10: Preparación del documento final \\
Tarea 11: Preparación de la presentación \\


\begin{table}[h]

\begin{tabular}{|c||c|c|c|c|c|c|c|c|c|c|c|c|c|c|c|c|}

\hline
Tarea / Semana & 1 & 2 & 3 & 4 & 5 & 6 & 7 & 8 & 9 & 10 & 11 & 12 & 13 & 14 & 15 & 16  \\
\hline\hline
1 & X & X & X & X &   &   &   &   &   &   &   &   &   &   &   &  \\
\hline
2 &   &  &  & X & X &   &   &   &   &   &   &   &   &   &   &  \\
\hline
3 &   &   &   &   & X & X &  &   &   &   &   &   &   &   &   &  \\
\hline
4 &   &   &   &   &   & X &  &  &   &   &   &   &   &   &   &  \\
\hline
5 &   &   &   &   &   & X & X &  &  &   &   &   &   &   &   &  \\
\hline
6 &   &   &   &   &   &   & X & X &   &   &   &   &   &   &   &  \\
\hline
7 &   &   &   &   &   &   &  & X & X & X &  &  &  &   &   &  \\
\hline
8 &   &   &   &   &   &   &   &   &   & X & X & X &  &   &   &  \\
\hline
9 &   &   &   &   &   &   &   &   &   &   &   & X & X &  &  &  \\
\hline
10 &   &   &   &   &   &   &   &   &   &   &   &   & X & X & X & \\
\hline
11 &   &   &   &   &   &   &   &   &   &   &   &   &   &   &  & X\\
\hline


\end{tabular}

\end{table}

\section{Personas Conocedoras del Tema}

\begin{itemize}

\item Jaime Forero (Universidad de los Andes)
\item Pedro Bargueño (Universidad de los Andes)
\item Nelson Padilla(Pontificia Universidad Católica de Chile)

\end{itemize}


\begin{thebibliography}{9}

\bibitem{Gadget}
Springel, Vorkel,"GADGET-2 A code for cosmological simulations of structure formation", Cosmological simulations with GADGET, November 05 2015
http://wwwmpa.mpa-garching.mpg.de/gadget/

\bibitem{Forero-Romero}
J.E. Forero-Romero, Y. Hoffman, S. Gottloeber, A. Klypin, G. Yepes, A Dynamical Classification of the Cosmic Web, arXiv:0809.4135 [astro-ph].

\bibitem{springel}
V. Springel, The cosmological simulation code GADGET-2, Mon. Not.R.Astron. Soc.364,1105–1134 (2005)       

\bibitem{tv-web}
https://github.com/astroandes/TV-Web


\end{thebibliography}

Firma del Director


Firma del Codirector





\end{document}
