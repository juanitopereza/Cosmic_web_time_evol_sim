\documentclass{article}
\usepackage[utf8]{inputenc}
\usepackage{amsmath}

\title{Evoluci\'on temporal de la red c\'osmica}
\author{Juan Sebastián Pérez\\ \\Director: Jaime Forero}
\date{Noviembre 2015}

\begin{document}

\maketitle

\section{Introducción}

Las simulaciones cosmológicas han sido de gran utilidad en las últimas
décadas para predecir la distribuci\'on de materia en el universo a
grandes escalas.  Los resultados de simulaciones pueden ser
comparados directamente con las observaciones astron\'omicas de la
distribuci\'on de galaxias.

Tanto las observaciones astronómicas como las simulaciones muestran
una estructura cosmológica bien definida la cual se conoce como
\emph{red c\'osmica}, la cual cuenta con elementos estructurales
definidos según sus características geométricas como vacíos,
filamentos, hojas y nudos.  Así pues, el estudio de las
características morfológicas de esta red son un tema de estudio de
alto impacto en la astrofísica moderna.

En este trabajo se har\'a una caracterización de la red c\'osmica a en
simulaciones cosmológicas de N-cuerpos.  El objetivo principal es el
estudio de la evoluci\'on de la red c\'osmica en escalas de tiempo
cosmol\'ogicas.  La simulación cosmol\'ogica se hará con el c\'odigo
Gadget [1], un estándar para la realización de simulaciones
cosmológicas. GADGET est\'a desarrollado para trabajar en sistemas de
cómputo de múltiples procesadores en paralelo,


Los datos obtenidos en la simulación se analizar\'an para caracterizar
la red c\'osmica en el volumen simulado.  El an\'alisis utiliza los
valores propios de $T_{\alpha \beta}$, i.e, el Hessiano del potencial
gravitacional $\phi$, para clasificar localmente cada lugar de la
simulaci\'on como un vac\'io, filamento, hoja o nudo [2][3].


\section{Objetivo General}

Realizar una caracterización de los elementos estructurales de la
cosmic web (vacíos, filamentos, hojas y nudos) y su evolución temporal
a partir de los datos generados por una simulación cosmológica
realizada con el código GADGET-2.  


\section{Objetivos Específicos}

\begin{enumerate}


\item Realizar simulacines cosmológicas usando el código GADGET-2.
\item Identificar los elementos de la red c\'osmica en una simulaci\'on
  cosmol\'ogica.
\item Analizar la evolución de la estructura observada para la red
  c\'osmica en cada paso temporal de la simulaci\'on. 
\end{enumerate}


\section{Metodología}


Esta será una monografía computacional. Utilizar\'e el código
GADGET2 qpara correr simulaciones cosmológicas en máquinas de
múltiples procesadores paralelos, por lo tanto, es necesario tener
acceso al cluster de la universidad como primera
medida. Adicionalmente, y aunque no es indispensable, se requerirá
acceso al laboratorio computacional de ciencias ubicado enel salón
Q406.  
  
La primera etapa del trabajo se centrará en familiarizarse con las
herramientas tecnológicas necesarias para la elaboración de la
simulación. Lo anterior corresponderá a familiarizarse con el manejo
de sistemas de cómputo de múltiples procesadores en paralelo, en
específico, obtener acceso al cluster de la universidad y a manejar su
ambiente de programación. Adicionalmente, en esta etapa se pretende
obtener un conocimiento básico del funcionamiento del código GADGET2
que será usado para la simulación cosmológica a partir de la cual se
generarán los datos. Para familiarizarse con el programa, se harán
simulaciones sencillas las cuales vienen incluidas en la documentación
del código, tales como colisiones de galaxias, colapso de nubes de
gas, entre otras.  

Posteriormente, y una vez se haya obtenido un entendimiento adecuado
del funcionamiento de GADGET2, se procederá a realizar la simulación
cosmológica necesaria para el análisis estructural de los elementos de
la cosmic web. Como se mencionó anteriormente, el análisis consistirá
en calcular los valores propios del Hessiano del potencial
gravitacional evaluado en cada punto espacial con el fin de determinar
las características geométricas de cada uno de ellos y así
clasificarlos como vacíos, filamentos, hojas o nudos. 

Finalmente, se realizará el análisis comentado anteriormente para cada
paso temporal con propósito de tener una imagen de la evolución
temporal de estas estructuras y de su aporte a la morfológica de la
cosmic web. En esta última etapa se realizarán comparaciones con datos
de otras simulaciones y de observaciones astronómicas con el fin de
corroborar que la estructura obtenida en este estudio corresponde con
la estructura cosmológica observada.          



\section{Cronograma}

Tarea 1: Revisión inicial de la literatura \\ 
Tarea 2: Familiarización con el manejo del cluster y primeros
acercamientos al uso de GADGET-2 \\ 
Tarea 3: Correr las simulaciones de prueba que vienen incluidas en la
documentación de GADGET-2  \\ 
Tarea 4: Investigar acerca de los parámetros que se usarán para la elaboración de la simulación que generará los datos \\
Tarea 5: Realizar la simulación mencionada anteriormente \\
Tarea 6: Preparacion de primera entrega del documento y presentacion respectiva \\
Tarea 7: Analizar preliminarmente los datos obtenidos a partir de la
simulación \\ 
Tarea 8: Categorizar cada punto espacial de la simulación realizada
según sus características geométricas en vacíos, filamentos, hojas o
nudos (Diagonalización de $T_{\alpha\beta}$). \\ 
Tarea 9: Estudiar la evolución temporal de las características
geométricas de cada punto espacial. \\ 
Tarea 10: Interpretar los resultados obtenidos de la caracterización para obtener así información acerca de la estructura morfológica de la cosmic web \\
Tarea 11: Preparación del documento final \\



\begin{table}[h]

\begin{tabular}{|c||c|c|c|c|c|c|c|c|c|c|c|c|c|c|c|c|}

\hline
Tarea / Semana & 1 & 2 & 3 & 4 & 5 & 6 & 7 & 8 & 9 & 10 & 11 & 12 & 13 & 14 & 15 & 16  \\
\hline\hline
1 & X & X & X &  &   &   &   &   &   &   &   &   &   &   &   &  \\
\hline
2 &   &  & X & X &  &   &   &   &   &   &   &   &   &   &   &  \\
\hline
3 &   &   &  X & X  & & &  &   &   &   &   &   &   &   &   &  \\
\hline
4 &   &   &   &  X & X  &  &  &  &   &   &   &   &   &   &   &  \\
\hline
5 &   &   &   &   & X  & X & X &  X &  &   &   &   &   &   &   &  \\
\hline
6 &   &   &   &   &   & X  & X & X &   &   &   &   &   &   &   &  \\
\hline
7 &   &   &   &   &   & X & X & X & X &  &  &  &  &   &   &  \\
\hline
8 &   &   &   &   &   &   &   &   &   & X & X & X &  &   &   &  \\
\hline
9 &   &   &   &   &   &   &   &   &   &   & X  & X & X & X &  &  \\
\hline
10 &   &   &   &   &   &   &   &   &   &   &   & X & X & X & X & X\\
\hline



\end{tabular}

\end{table}

\section{Personas Conocedoras del Tema}

\begin{itemize}

\item Jaime Forero (Universidad de los Andes)
\item Pedro Bargueño (Universidad de los Andes)
\item Nelson Padilla(Pontificia Universidad Católica de Chile)

\end{itemize}


\begin{thebibliography}{9}

\bibitem{Gadget}
Springel, Vorkel,"GADGET-2 A code for cosmological simulations of structure formation", Cosmological simulations with GADGET, November 05 2015
http://wwwmpa.mpa-garching.mpg.de/gadget/

\bibitem{Forero-Romero}
J.E. Forero-Romero, Y. Hoffman, S. Gottloeber, A. Klypin, G. Yepes, A Dynamical Classification of the Cosmic Web, ArXiv e-prints, astro-ph 0809.4135

\bibitem{tv-web}
https://github.com/astroandes/TV-Web


\end{thebibliography}

Firma del Director


Firma del Codirector





\end{document}
