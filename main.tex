\documentclass{article}
\usepackage[utf8]{inputenc}
\usepackage{amsmath}

\title{Propuesta Monografia}
\author{Juan Sebastián Pérez\\ \\Director: Jaime Forero}
\date{November 2015}

\begin{document}

\maketitle

\section{Introducción}

Las simulaciones cosmológicas han sido de gran utilidad en 
las últimas décadas para predecir la distribuci\'on de materia 
en el universo a grandes escalas.
Los resultados des simulaciones pueden ser comparados
directamente con las observaciones astron\'omicas de la distribuci\'on
de galaxias. 
Tanto las observaciones astronómicas como las simulaciones muestran
una estructura cosmológica bien definida la cual se conoce como
“cosmic web”, la cual cuenta con elementos 
estructurales definidos según sus características geométricas como
vacíos, filamentos, hojas y nudos.  
Así pues, el estudio de las  características morfológicas de esta red
son un tema de estudio de alto impacto en la astrofísica moderna. 

En este trabajo se har\'a una caracterización de la red c\'osmica a
en simulaciones cosmológicas de N-cuerpos. 
El objetivo principal es el estudio de la evoluci\'on de la red
c\'osmica en escalas de tiempo cosmol\'ogicas.
La simulación cosmol\'ogica se hará con el c\'odigo Gadget [REF], 
un estándar para la realización de simulaciones cosmológicas. GADGET
est\'a desarrollado para trabajar en sistemas de cómputo de múltiples
procesadores en paralelo, 


Los datos obtenidos en la simulación se analizar\'an para caracterizar
la red c\'osmica en el volumen simulado. 
El an\'alisis  utiliza los valores propios del Hessiano del potencial
gravitacional $\phi$ para clasificar localmente cada lugar de la
simulaci\'on como un vac\'io, filamento, hoja o nudo.
[ESTE CODIGO TAMBIEN ES PUBLICO \verb"https://github.com/astroandes/TV-web"]


\section{Objetivo General}

Realizar una caracterización de los elementos estructurales de la
cosmic web (vacíos, filamentos, hojas y nudos) y su evolución temporal
a partir de los datos generados por una simulación cosmológica
realizada con el código GADGET2. 

\section{Objetivos Específicos}

\begin{enumerate}

\item Familiarizarse con el uso de máquinas de c\'omputo de múltiples
  procesadores en paralelo.  
\item Aprender a compilar y correr el código GADGET2 para simulaciones
  de N-cuerpos. 
\item Realizar una simulación cosmológica, usando los parámetros
  cosmológicos definidos por las últimas observaciones astronómicas y
  usando el código GADGET2. 
\item Evaluar el Hessiano del potencial gravitacional para cada punto
  espacial y en cada paso temporal y encontrar los valores propios que
  se encuentran por encima de un cierto umbral. 
\item Clasificar cada punto espacial, dependiendo de los valores
  propios obtenidos, como: vacíos, filamentos, hojas o nudos. 
\item Analizar la evolución de la estructura observada en cada paso
  temporal. 
\end{enumerate}


\section{Metodología}

Esta será una monografía computacional. Utilizar\'e el código
GADGET2 qpara correr simulaciones cosmológicas en máquinas de
múltiples procesadores paralelos, por lo tanto, es necesario tener
acceso al cluster de la universidad como primera
medida. Adicionalmente, y aunque no es indispensable, se requerirá
acceso al laboratorio computacional de ciencias ubicado enel salón
Q406.  
  
La primera etapa del trabajo se centrará en familiarizarse con las
herramientas tecnológicas necesarias para la elaboración de la
simulación. Lo anterior corresponderá a familiarizarse con el manejo
de sistemas de cómputo de múltiples procesadores en paralelo, en
específico, obtener acceso al cluster de la universidad y a manejar su
ambiente de programación. Adicionalmente, en esta etapa se pretende
obtener un conocimiento básico del funcionamiento del código GADGET2
que será usado para la simulación cosmológica a partir de la cual se
generarán los datos. Para familiarizarse con el programa, se harán
simulaciones sencillas las cuales vienen incluidas en la documentación
del código, tales como colisiones de galaxias, colapso de nubes de
gas, entre otras.  

Posteriormente, y una vez se haya obtenido un entendimiento adecuado
del funcionamiento de GADGET2, se procederá a realizar la simulación
cosmológica necesaria para el análisis estructural de los elementos de
la cosmic web. Como se mencionó anteriormente, el análisis consistirá
en calcular los valores propios del Hessiano del potencial
gravitacional evaluado en cada punto espacial con el fin de determinar
las características geométricas de cada uno de ellos y así
clasificarlos como vacíos, filamentos, hojas o nudos. 

Finalmente, se realizará el análisis comentado anteriormente para cada
paso temporal con propósito de tener una imagen de la evolución
temporal de estas estructuras y de su aporte a la morfológica de la
cosmic web. En esta última etapa se realizarán comparaciones con datos
de otras simulaciones y de observaciones astronómicas con el fin de
corroborar que la estructura obtenida en este estudio corresponde con
la estructura cosmológica observada.          


\section{Cronograma}

Tarea 1: Descripción de la tarea 1 \\
Tarea 2: Descripción de la tarea 2 \\
Tarea 3: Descripción de la tarea 3 \\

\begin{table}[htb]

\begin{tabular}{|c||c|c|c|c|c|c|c|c|c|c|c|c|c|c|c|c| }

\hline
Tarea / Semana & 1 & 2 & 3 & 4 & 5 & 6 & 7 & 8 & 9 & 10 & 11 & 12 & 13 & 14 & 15 & 16  \\
\hline\hline
1 &   &   &   &   &   &   &   &   &   &   &   &   &   &   &   &  \\
\hline
2 &   &   &   &   &   &   &   &   &   &   &   &   &   &   &   &  \\
\hline
3 &   &   &   &   &   &   &   &   &   &   &   &   &   &   &   &  \\
\hline
4 &   &   &   &   &   &   &   &   &   &   &   &   &   &   &   &  \\
\hline
5 &   &   &   &   &   &   &   &   &   &   &   &   &   &   &   &  \\
\hline
6 &   &   &   &   &   &   &   &   &   &   &   &   &   &   &   &  \\
\hline
7 &   &   &   &   &   &   &   &   &   &   &   &   &   &   &   &  \\
\hline
8 &   &   &   &   &   &   &   &   &   &   &   &   &   &   &   &  \\
\hline

\end{tabular}

\end{table}

\section{Personas Conocedoras del Tema}

\begin{itemize}

\item Jaime Forero (Universidad de los Andes)
\item Pedro Bargueño (Universidad de los Andes)
\item jj

\end{itemize}


\section{Referencias}

[1] 


Firma del Director


Firma del Codirector





\end{document}
