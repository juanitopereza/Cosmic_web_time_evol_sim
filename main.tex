\documentclass{article}
\usepackage[utf8]{inputenc}
\usepackage{amsmath}

\title{Propuesta Monografia}
\author{Juan Sebastián Pérez\\ \\Director: Jaime Forero}
\date{November 2015}

\begin{document}

\maketitle

\section{Introducción}

Las simulaciones cosmológicas han resultado ser de gran utilidad en las últimas décadas como un mecanismo para realizar predicciones acerca de la estructura del universo tanto a gran escala como a pequeña escala, permitiendo así obtener resultados que pueden ser sometidos a comparación con las observaciones astronómicas realizadas. Uno de los resultados más importantes acerca del análisis de esta estructura es que tanto las observaciones astronómicas como las simulaciones, muestran una estructura cosmológica bien definida, la cual se conoce como “cosmic web”, la cual cuenta con elementos estructurales definidos según sus características geométricas, siendo estos: vacíos, filamentos, hojas y nudos. Así pues, el estudio de las características morfológicas de esta red y de su evolución temporal han resultado un tema de estudio de alto impacto en la astrofísica moderna.

Teniendo en cuenta lo anterior, en este trabajo se pretende realizar una caracterización de los elementos estructurales de la cosmic web nombrados anteriormente, a partir del análisis de los datos obtenidos al realizar una simulación cosmológica de N-cuerpos. 

Para crear la simulación, se hará uso del código GADGET2, creado por el profesor Dr. Volker Springel de la universidad de Heildeberg, Alemania. Este código se ha convertido en un estándar para la realización de simulaciones cosmológicas y fue desarrollado para trabajar en sistemas de cómputo de múltiples procesadores en paralelo, lo cual permite optimizar el desempeño del programa para así lograr simulaciones con un número de partículas considerable.

Posteriormente, se usarán los datos obtenidos a partir de la simulación para realizar un análisis de las características geométricas de cada punto espacial con el fin de categorizarlos en 4 grupos: vacíos, filamentos, hojas y nudos. Para realizar esta categorización, se estudiarán los valores propios del tensor de deformación, el cual corresponde al Hessiano del potencial gravitacional $\phi$. Para realizar la clasificación, se diagonal izará el Hessiano del potencial gravitacional evaluado en cada punto espacial y se hará una cuenta de los valores propios obtenidos por encima de cierto umbral; si el número de valores propios obtenidos es cero, el punto corresponderá a un vacío, si el número es uno corresponderá a un filamento, si es dos será una hoja y si es tres será un nudo.    

\section{Objetivo General}

Realizar una caracterización de los elementos estructurales de la cosmic web (vacíos, filamentos, hojas y nudos) y su evolución temporal a partir de los datos generados por una simulación cosmológica realizada con el código GADGET2.

\section{Objetivos Específicos}

\begin{enumerate}

\item Familiarizarse con el uso de máquinas de computo de múltiples procesadores en paralelo. 
\item Aprender a compilar y correr el código GADGET2 para simulaciones de N-cuerpos.
\item Realizar una simulación cosmológica, usando los parámetros cosmológicos definidos por las últimas observaciones astronómicas y usando el código GADGET2.
\item Evaluar el Hessiano del potencial gravitacional para cada punto espacial y en cada paso temporal y encontrar los valores propios que se encuentran por encima de un cierto umbral.
\item Clasificar cada punto espacial, dependiendo de los valores propios obtenidos, como: vacíos, filamentos, hojas o nudos.
\item Analizar la evolución de la estructura observada en cada paso temporal.

\end{enumerate}


\section{Metodología}

Esta será una monografía computacional en la cuál se usará el código GADGET2 que ha sido desarrollado  para correr simulaciones cosmológicas en máquinas de múltiples procesadores paralelos, por lo tanto, es necesario tener acceso al cluster de la universidad como primera medida. Adicionalmente, y aunque no es indispensable, se requerirá acceso al laboratorio computacional de ciencias ubicado en el salón Q406.

La primera etapa del trabajo se centrará en familiarizarse con las herramientas tecnológicas necesarias para la elaboración de la simulación. Lo anterior corresponderá a familiarizarse con el manejo de sistemas de cómputo de múltiples procesadores en paralelo, en específico, obtener acceso al cluster de la universidad y a manejar su ambiente de programación. Adicionalmente, en esta etapa se pretende obtener un conocimiento básico del funcionamiento del código GADGET2 que será usado para la simulación cosmológica a partir de la cual se generarán los datos. Para familiarizarse con el programa, se harán simulaciones sencillas las cuales vienen incluidas en la documentación del código, tales como colisiones de galaxias, colapso de nubes de gas, entre otras. 

Posteriormente, y una vez se halla obtenido un entendimiento adecuado del funcionamiento de GADGET2, se procederá a realizar la simulación cosmológica necesaria para el análisis estructural de los elementos de la cosmic web. Como se mencionó anteriormente, el análisis consistirá en calcular los valores propios del Hessiano del potencial gravitacional evaluado en cada punto espacial con el fin de determinar las características geométricas de cada uno de ellos y así clasificarlos como vacíos, filamentos, hojas o nudos.

Finalmente, se realizará el análisis comentado anteriormente para cada paso temporal con propósito de tener una imagen de la evolución temporal de estas estructuras y de su aporte a la morfológica de la cosmic web. En esta última etapa se realizarán comparaciones con datos de otras simulaciones y de observaciones astronómicas con el fin de corroborar que la estructura obtenida en este estudio corresponde con la estructura cosmológica observada.         


\section{Cronograma}

Tarea 1: Descripción de la tarea 1 \\
Tarea 2: Descripción de la tarea 2 \\
Tarea 3: Descripción de la tarea 3 \\

\begin{table}[htb]

\begin{tabular}{|c||c|c|c|c|c|c|c|c|c|c|c|c|c|c|c|c| }

\hline
Tarea / Semana & 1 & 2 & 3 & 4 & 5 & 6 & 7 & 8 & 9 & 10 & 11 & 12 & 13 & 14 & 15 & 16  \\
\hline\hline
1 &   &   &   &   &   &   &   &   &   &   &   &   &   &   &   &  \\
\hline
2 &   &   &   &   &   &   &   &   &   &   &   &   &   &   &   &  \\
\hline
3 &   &   &   &   &   &   &   &   &   &   &   &   &   &   &   &  \\
\hline
4 &   &   &   &   &   &   &   &   &   &   &   &   &   &   &   &  \\
\hline
5 &   &   &   &   &   &   &   &   &   &   &   &   &   &   &   &  \\
\hline
6 &   &   &   &   &   &   &   &   &   &   &   &   &   &   &   &  \\
\hline
7 &   &   &   &   &   &   &   &   &   &   &   &   &   &   &   &  \\
\hline
8 &   &   &   &   &   &   &   &   &   &   &   &   &   &   &   &  \\
\hline

\end{tabular}

\end{table}

\section{Personas Conocedoras del Tema}

\begin{itemize}

\item Jaime Forero (Universidad de los Andes)
\item Pedro Bargueño (Universidad de los Andes)
\item jj

\end{itemize}


\section{Referencias}

[1] 


Firma del Director


Firma del Codirector





\end{document}
